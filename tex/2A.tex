\chapter{Lie algebra - general structure}
The main aim of this chapter will be to develop some contrasting notions of \emph{solvable} and \emph{semisimple} Lie algebras.
Informally, solvable Lie algebras are those that are ``nearly abelian'', which have uninteresting and unwieldy representations.
semisimple Lie algebras are those which are ``far from abelian'', which have interesting representations.
We will see later in chapter 4 that this second type are classifiable.

Note that Representation theory will always be in sight, as this is how Lie algebras show up in perspective, and that also in a sense, solvable and semisimple Lie algebras can cover everything as any Lie algebra can be viewed as a formation of solvable and semisimple building blocks.

\section{Basic definitions, subalgebras, ideals}

\begin{defn}
	A \emph{Lie algebra} $L$ is a vector space with a bilinear map \\ $[\cdot, \cdot] : L \times L \to L$, satisfying 
	\begin{easylist}[enumerate]
		& The anti-symmetric property
		\[ [x,y] = -[y,x] \qquad \forall x,y,z \in L \]
		& The Jacobi identity
		\[ [x,[y,z]] + [z,[x,y]] + [y,[z,x]] = 0 \qquad \forall x,y,z \in L \]
	\end{easylist}
\end{defn} 

Here we shall always take $L$ to be a finite-dimensional vector space over $\mathbb{R}$ or $\mathbb{C}$.
For now we will consider the Lie algebras Lie$(G)$ of Lie groups as $\mathbb{R}$-vector spaces.

\begin{defn}
	We say that a Lie algebra L is \emph{abelian} if $[\cdot,\cdot] = 0$
\end{defn}

Note that if $L$ has dimension 1 then $L$ is abelian, due to the property of anti-symmetry.

\begin{defn}
	For Lie algebras $L_1$ and $L_2$, a \emph{Lie algebra homomorphism} over $\mathbb{F}$ is an $\mathbb{F}$-linear map $\varphi : L_1 \to L_2$ such that
		\[ [\varphi(x), \varphi(y)]_{L_2} = \varphi([x,y]_{L_1}) \qquad \forall x,y \in L_1 \]
\end{defn}

\begin{defn}
	We call $\varphi$ an \emph{isomorphism} if is invertible and an \emph{automorphism} if it is an isomorphism and $L_1 = L_2$
\end{defn}

\begin{exmp}
	The endomorphism group of a vector space $V$
		\[ \text{End}(V) := \{f: V \to V \ | \ V \text{ is linear } \}\]
	is a Lie algebra over $\mathbb{F}$ with bracket equal to the commutator of linear maps, for any $\mathbb{F}$-vector space V.
	We also denote this gl$(V) = $ End$(V)$

	A special case for this occurs when $V = \mathbb{F}^n$, where gl$(V) = $ Mat$(n,\mathbb{F})$
\end{exmp}


\begin{exmp}
	A \emph{representation} of a Lie algebra $L$ on a vector space $V$ is a morphism $\varphi : L \to $ End$(V)$
\end{exmp}


\begin{defn}
	For a Lie algebra $L$, we call a subvector space $H \subset L$ a \emph{Lie-subalgebra} (or `sub-Lie algebra' or  `subalgebra') if $H$ is closed under the bracket, i.e.
		\[ [x,y] \in H \qquad \forall x,y \in H \]
	For which we write $[H,H] \subset H$.
\end{defn}



\begin{defn}
	An \emph{Ideal} $I$ of a Lie algebra $L$ is a subspace $I \subset L$ such that $[L,I] \subset I$. That is
		\[ [x,y] \in I \qquad \forall x \in L, \forall y \in I \]
	Note that any ideal is a subalgebra.
\end{defn}

\begin{lemm}
	If $I \subset L$ is an ideal; then the quotient space
		\[L/I := \{x + I \mid x \in L \} \]
	carries a canonical bracket
		\[[x+I, y+I] := [x,y] + I\]
	and $\pi : L \to L/I$ is surjective morphism of Lie algebras. Note that $x+I := \{x + z \mid z \in I \} \subset L$)
\end{lemm}

\begin{lemm}
If $\varphi : L_1 \to L_2$ is a morphism of Lie algebras then
\begin{easylist}[enumerate]	
& Ker$\varphi$ is an ideal in $L_1$
& Im$\varphi$ is a subalgebra of $L_2$
& $L_1/Ker\varphi$ is isomorphic to Im$\varphi$
\end{easylist}
\end{lemm}

\begin{lemm}
	if I, J ideals in L $\implies I,J, I+J, [I,J]$ are also ideals in $L$.
\end{lemm}

\begin{lemm}
	$(I + J)/J $ isomorphic $I/(I,J)$
\end{lemm}

See problem sheets for proof
